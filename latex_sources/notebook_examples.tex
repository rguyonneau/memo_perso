\section{Exemples \LaTeX pour l'utilisation du cahier}

Cette section présente des exemples pour l'utilisation de la classe notebook. 

\subsection{Écrire du code}

\subsubsection{Terminal}

Pour écrire une sortie de terminal, il y a l'environnement terminal. Par exemple, écrire:

\begin{TeXlstlisting}
\begin{terminal}
sudo apt install monpackage
\end{terminal}
\end{TeXlstlisting}

permet d'avoir le rendu:

\begin{terminal}
sudo apt install monpackage
\end{terminal}

\subsubsection{Code C++}

Pour écrire du code C++ il est possible d'utiliser l'environnement \textbf{lstlisting} avec l'option \textbf{C++}. Par exemple, écrire:

\begin{TeXlstlisting}
\begin{lstlisting}[language=C++]
// un commentaire
int main(void){
    printf("test");
}
\end{lstlisting}
\end{TeXlstlisting}

permet d'avoir le rendu :

\begin{lstlisting}[language=C++]
// un commentaire
int main(void){
    printf("test");
}
\end{lstlisting}

\subsubsection{Code Python}

Pour écrire du code Python il est possible d'utiliser l'environnement \textbf{lstlisting} avec l'option \textbf{Python}. Par exemple, écrire:
\begin{TeXlstlisting}
\begin{lstlisting}[language=Python]
# un commentaire
for i in range(0, 10):
    print(i)
\end{lstlisting}
\end{TeXlstlisting}

permet d'obtenir:

\begin{lstlisting}[language=Python]
# un commentaire
for i in range(0, 10):
    print(i)
\end{lstlisting}
