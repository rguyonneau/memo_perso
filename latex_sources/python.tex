\section{Notes sur le langage Python}

%%%%%%%%%%%%%%%%%%%%%%%%%%%%%%%%%%%%%%%%%%%%%%%%%%%%%%%%%%%%%%%%%%%%%%%
%                       Commandes utiles
%%%%%%%%%%%%%%%%%%%%%%%%%%%%%%%%%%%%%%%%%%%%%%%%%%%%%%%%%%%%%%%%%%%%%%%
\subsection{Commande utiles}

% ---------------------------------------------------------------------
\subsubsection{le terminal python}

Lancer l'interpreteur:

\begin{terminal}
python3
Python 3.6.9 (default, Apr 18 2020, 01:56:04) 
[GCC 8.4.0] on linux
Type "help", "copyright", "credits" or "license" for more information.
>>> 
\end{terminal}

Quitter l'interpréteur

\begin{terminal}
>>> exit()
\end{terminal}

% ---------------------------------------------------------------------
\subsubsection{La commande dir()}

La fonction dir() retourne toutes les propriétés et méthodes d'un objet donné en paramètre. Sans paramètre en entrée elle permet notamment d'accéder à tous les packages disponibles.

\begin{lstlisting}[language=Python]
dir()  # Lister tous les imports disponibles a ce moment du code
\end{lstlisting}

Exemple d'utilisation:

\begin{terminal}
>>> dir()
['__annotations__', '__builtins__', '__doc__', '__loader__', '__name__', '__package__', '__spec__']', '__package__', '__spec__']
>>> import tkinter
>>> dir()
['__annotations__', '__builtins__', '__doc__', '__loader__', '__name__', '__package__', '__spec__', 'tkinter']
\end{terminal}


\begin{lstlisting}[language=Python]
dir(name)  # Lister les proprietes/methodes disponibles pour "name"
\end{lstlisting}

Exemple d'utilisation:

\begin{terminal}
>>> dir(tkinter)
['ACTIVE', 'ALL', ...,  'Text', 'Tk', 'TkVersion', ...]
>>> dir(tkinter.Tk)
['_Misc__winfo_getint', '_Misc__winfo_parseitem', '__class__', ..., 'after', 'after_cancel', 'after_idle', 'anchor', 'aspect', 'attributes', 'bbox', ...]
\end{terminal}

% ---------------------------------------------------------------------
\subsubsection{La commande help()}

Elle permet d'accéder à l'aide python. Elle prend un argument, qui correspond à l'objet pour lequel on veut obtenir des informations.
\begin{lstlisting}[language=Python]
help(name)  # Obtenir de l aide sur le module "name"
\end{lstlisting}

Exemple d'utilisation:

\begin{terminal}
>>> help(tkinter.Tk)
>>> help(list)
\end{terminal}

Pour quitter l'aide on peut utiliser la touche 'q' du clavier.

% ---------------------------------------------------------------------
\subsubsection{Gestion de grands nombres}

Il est possible d'utiliser l'underscore pour améliorer la visibilité de grands nombres:
\begin{lstlisting}[language=Python]
un_grand_nombre = 10_000_000_000
un_plus_petit_grand_nombre = 2_000_000_000
somme = un_grand_nombre + un_plus_petit_grand_nombre
print(f"{somme:_}")
\end{lstlisting}
Ce qui donne l'affichage suivant
\begin{terminal}
12_000_000_000
\end{terminal}

\subsubsection{La commande type()}

Il est possible d'utiliser la commande type() pour connaître le type d'une variable. Par exemple

\begin{lstlisting}[language=Python]
variables = [1, 3.2, lambda x: x + 1, (1, 2), [1, 2]]
for var in variables:
    print(type(var))
\end{lstlisting}
Donne l'affichage suivant
\begin{terminal}
<class 'int'>
<class 'float'>
<class 'function'>
<class 'tuple'>
<class 'list'>
\end{terminal}


\subsubsection{Les variables non utilisées}

La convention en python pour la gestion des variables non utilisées est de les appeler "\_". Des exemple:

\begin{lstlisting}[language=Python]
mon_tuple = (1, 2, 3)
val1, _, val2 = mon_tuple
print(f"val1:{val1}, val2:{val2}")
\end{lstlisting}
Affiche
\begin{terminal}
val1:1, val2:3
\end{terminal}
La deuxième valeur du tuple n'est pas utilisée.


% ---------------------------------------------------------------------
\subsubsection{todo?}

pip list

which python

%%%%%%%%%%%%%%%%%%%%%%%%%%%%%%%%%%%%%%%%%%%%%%%%%%%%%%%%%%%%%%%%%%%%%%%
%                     Environnements virtuels
%%%%%%%%%%%%%%%%%%%%%%%%%%%%%%%%%%%%%%%%%%%%%%%%%%%%%%%%%%%%%%%%%%%%%%%
\subsection{Environnements virtuels}

% ---------------------------------------------------------------------
\subsubsection{Pourquoi?}
Un espace pour installer des paquets spécifiques à un ou plusieurs projets

% ---------------------------------------------------------------------
\subsubsection{Créer un environnement avec venv (standard)}

\begin{terminal}
python3 -m venv test_env
\end{terminal}
Remarque: le "-m" c'est pour dire qu'on lance le module venv. Le module venv attend un argument: le nom de l'environnement (test\_env) ici.

% ---------------------------------------------------------------------
\subsubsection{Activer l'environnement}

\begin{terminal}
source test_env/bin/activate
\end{terminal}

%%%%%%%%%%%%%%%%%%%%%%%%%%%%%%%%%%%%%%%%%%%%%%%%%%%%%%%%%%%%%%%%%%%%%%%
%                       Listes
%%%%%%%%%%%%%%%%%%%%%%%%%%%%%%%%%%%%%%%%%%%%%%%%%%%%%%%%%%%%%%%%%%%%%%%
\subsection{Listes}

% ---------------------------------------------------------------------
\subsubsection{Accéder à des éléments}

Pour accéder aux derniers éléments d'une liste:

\begin{lstlisting}[language=Python]
ma_liste = [1,2,3,4]
print(ma_liste[-1])  # affiche 4
print(ma_liste[-2])  # affiche 3
\end{lstlisting}

Attention, cette notation aussi peut générer une exception "IndexError".

% ---------------------------------------------------------------------
\subsubsection{Parcourir des listes}

Pour parcourir une liste sans utiliser d'indice
\begin{lstlisting}[language=Python]
ma_liste = [1, 2,"test"]
for val in ma_liste:
    print(val)
\end{lstlisting}
Affiche:
\begin{terminal}
1
2
test
\end{terminal}

Pour parcourir une liste tout en comptant les éléments:
\begin{lstlisting}[language=Python]
ma_liste = [1, 2, "test"]
for idx, val in enumerate(ma_liste):
    print(f"ma_liste[{idx}] = {val}")
\end{lstlisting}
Affiche:
\begin{terminal}
ma_liste[0] = 1
ma_liste[1] = 2
ma_liste[2] = test
\end{terminal}

Il est aussi possible de commencer à compter à partir d'une certaine valeure:
\begin{lstlisting}[language=Python]
ma_liste = [1, 2, "test"]
for idx, val in enumerate(ma_liste, start=1):
    print(f"{idx} : {val}")
\end{lstlisting}
Affiche:
\begin{terminal}
1 : 1
2 : 2
3 : test
\end{terminal}

% ---------------------------------------------------------------------
\subsubsection{Parcourir deux listes en même temps}

Il est possible d'utiliser la fonction zip() pour parcourir plusieurs listes. Un exemple avec deux listes, mais peut être étendu à plus de listes:

\begin{lstlisting}[language=Python]
list1 = [1, 2, 3]
list2 = ["un", "deux", "trois"]

for val1, val2 in zip(list1, list2):
    print(f"{val1} : {val2}")
\end{lstlisting}

\begin{terminal}
1 : un
2 : deux
3 : trois
\end{terminal}

%%%%%%%%%%%%%%%%%%%%%%%%%%%%%%%%%%%%%%%%%%%%%%%%%%%%%%%%%%%%%%%%%%%%%%%
%                       Tuple
%%%%%%%%%%%%%%%%%%%%%%%%%%%%%%%%%%%%%%%%%%%%%%%%%%%%%%%%%%%%%%%%%%%%%%%
\subsection{Tuples}

\subsubsection{Récupérer les valeurs}

Récupérer toutes les valeurs:
\begin{lstlisting}[language=Python]
mon_tupple = (1, 2, 3)
a, b, c = mon_tupple
print(f"a:{a}, b:{b}, c:{c}")
\end{lstlisting}

\begin{terminal}
a:1, b:2, c:3
\end{terminal}

Récupérer une partie des valeurs d'un tuple. Ce qu'il ne faut pas faire:
\begin{lstlisting}[language=Python]
mon_tupple = (1, 2, 3, 4, 5)
a, b, c = mon_tupple
print(f"a:{a}, b:{b}, c:{c}")
\end{lstlisting}
\begin{terminal}
    a, b, c = mon_tupple
ValueError: too many values to unpack (expected 3)
\end{terminal}
Ce qu'il est possible de faire:
\begin{lstlisting}[language=Python]
mon_tupple = (1, 2, 3, 4, 5)
a, b, *c = mon_tupple
print(f"a:{a}, b:{b}, c:{c}")
\end{lstlisting}

\begin{terminal}
a:1, b:2, c:[3, 4, 5]
\end{terminal}



%%%%%%%%%%%%%%%%%%%%%%%%%%%%%%%%%%%%%%%%%%%%%%%%%%%%%%%%%%%%%%%%%%%%%%%
%                         Saisies
%%%%%%%%%%%%%%%%%%%%%%%%%%%%%%%%%%%%%%%%%%%%%%%%%%%%%%%%%%%%%%%%%%%%%%%
\subsection{Saisies}

% ---------------------------------------------------------------------
\subsubsection{Récupérer une saisie sans l'afficher à l'écran (terminal)}

Cas d'utilisation:
\begin{lstlisting}[language=Python]
nom = input("Utilisateur: ")
psw = input("Mot de passe: ")
print("Connexion...")
\end{lstlisting}
\begin{terminal}
Utilisateur: remy
Mot de passe: mdp
Connexion...
\end{terminal}
On voit que le mot de passe s'affiche sur le terminal, ce qui n'est pas désirable... 

À la place, il est possible de faire:
\begin{lstlisting}[language=Python]
from getpass import getpass
nom = input("Utilisateur: ")
psw = getpass("Mot de passe: ")
print("Connexion...")
\end{lstlisting}
\begin{terminal}
Utilisateur: remy
Mot de passe:
Connexion...
\end{terminal}
Ce qui est quand même mieux...


%%%%%%%%%%%%%%%%%%%%%%%%%%%%%%%%%%%%%%%%%%%%%%%%%%%%%%%%%%%%%%%%%%%%%%%
%                         Classes
%%%%%%%%%%%%%%%%%%%%%%%%%%%%%%%%%%%%%%%%%%%%%%%%%%%%%%%%%%%%%%%%%%%%%%%
\subsection{Classes}

% ---------------------------------------------------------------------
\subsubsection{Définir un attribut dynamiquement}

Cas simple sans attribution dynamique du nom des attributs:
\begin{lstlisting}[language=Python]
# declaration d une classe "vide"
class Ma_classe():
    pass
# instanciation de la classe dans l objet inst
inst = Ma_classe()
# on ajoute un attribut "att1" a la classe et lui affecte la valeur "coucou"
inst.att1 = "coucou"
# on ajoute un attribut "att2" a la classe et lui affecte la valeur "coucou"
inst.att2 = 3
# affichage des valeurs des attributs
print(inst.att1, inst.att2)
\end{lstlisting}
\begin{terminal}
coucou 3
\end{terminal}

Le même exemple mais avec une attribution dynamique des noms des attributs
\begin{lstlisting}[language=Python]
# declaration d une classe "vide"
class Ma_classe():
    pass

# instanciation de la classe dans l objet inst
inst = Ma_classe()
# on passe par des variables pour les noms des attributs
att1_name = "att1"
att2_name = "att2"
# on ajoute un attribut "att1" a la classe et lui affecte la valeur "coucou"
setattr(inst, att1_name, "coucou")  # objet, nom, valeur
# on ajoute un attribut "att2" a la classe et lui affecte la valeur 3
setattr(inst, att2_name, 3)  # objet, nom, valeur
# affichage des valeurs des attributs
print(inst.att1, inst.att2)
\end{lstlisting}
\begin{terminal}
coucou 3
\end{terminal}

Un autre exemple d'utilisation:
\begin{lstlisting}[language=Python]
# declaration d une classe "vide"
class Ma_classe():
    pass

# instanciation de la classe dans l objet inst
inst = Ma_classe()

# defintion des attributs dans un dictionnaire
mes_attributs = {"att1":"coucou", "att2":3}

# ajout des attributs a la classe
for key, value in mes_attributs.items():
    setattr(inst, key, value)

# affichage des valeurs des attributs
print(inst.att1, inst.att2)
\end{lstlisting}
\begin{terminal}
coucou 3
\end{terminal}

\begin{lstlisting}[language=Python]
# declaration d une classe "vide"
class Ma_classe():
    pass

# instanciation de la classe dans l objet inst
inst = Ma_classe()

# defintion des attributs dans un dictionnaire
mes_attributs = {"att1":"coucou", "att2":3}

# ajout des attributs a la classe
for key, value in mes_attributs.items():
    setattr(inst, key, value)

# recuperation et affichage des attributs
for key in mes_attributs.keys():
    print(getattr(inst, key))
\end{lstlisting}
\begin{terminal}
coucou
3
\end{terminal}
%%%%%%%%%%%%%%%%%%%%%%%%%%%%%%%%%%%%%%%%%%%%%%%%%%%%%%%%%%%%%%%%%%%%%%%
%                     Gestion de fichiers/ressources
%%%%%%%%%%%%%%%%%%%%%%%%%%%%%%%%%%%%%%%%%%%%%%%%%%%%%%%%%%%%%%%%%%%%%%%
\subsection{Gestion de fichiers/ressources}

% ---------------------------------------------------------------------
\subsubsection{Context Manager}

Dès qu'une ressource demande une ouverture puis une fermeture (e.g. des fichiers), il est possible d'utiliser un context manager pour ne pas avoir à gérer la fermeture.


Par exemple, à la place de faire:

\begin{lstlisting}[language=Python]
f = open("tmp.txt", "r")
file_contents = f.read()
f.close()
print(file_contents)
\end{lstlisting}

\begin{terminal}
Contenu du fichier tmp.txt
Sur plusieurs lignes...
\end{terminal}

Il est possible de faire (ce qui donne le même résultat qu'au dessus)
\begin{lstlisting}[language=Python]
with open("tmp.txt", "r") as f:
    file_contents = f.read()
print(file_contents)
\end{lstlisting}

Le context manager peut être utilisé dès qu'il faut ouvrir et fermer des ressources (pour un lock dans un thread par exemple, pour l'accès à des base de données...).

%%%%%%%%%%%%%%%%%%%%%%%%%%%%%%%%%%%%%%%%%%%%%%%%%%%%%%%%%%%%%%%%%%%%%%%
%                     Lambda fonction
%%%%%%%%%%%%%%%%%%%%%%%%%%%%%%%%%%%%%%%%%%%%%%%%%%%%%%%%%%%%%%%%%%%%%%%
\subsection{Fonctions Lambda / Fonctions sans noms}




