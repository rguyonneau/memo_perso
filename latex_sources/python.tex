\section{Notes sur le langage Python}

%%%%%%%%%%%%%%%%%%%%%%%%%%%%%%%%%%%%%%%%%%%%%%%%%%%%%%%%%%%%%%%%%%%%%%%
%                       Commandes utiles
%%%%%%%%%%%%%%%%%%%%%%%%%%%%%%%%%%%%%%%%%%%%%%%%%%%%%%%%%%%%%%%%%%%%%%%
\subsection{Commande utiles}

% ---------------------------------------------------------------------
\subsubsection{le terminal python}

Lancer l'interpreteur:

\begin{terminal}
python3
Python 3.6.9 (default, Apr 18 2020, 01:56:04) 
[GCC 8.4.0] on linux
Type "help", "copyright", "credits" or "license" for more information.
>>> 
\end{terminal}

Quitter l'interpréteur

\begin{terminal}
>>> exit()
\end{terminal}

% ---------------------------------------------------------------------
\subsubsection{La commande dir()}

La fonction dir() retourne toutes les propriétés et méthodes d'un objet donné en paramètre. Sans paramètre en entrée elle permet notamment d'accéder à tous les packages disponibles.

\begin{lstlisting}[language=Python]
dir()  # Lister tous les imports disponibles a ce moment du code
\end{lstlisting}

Exemple d'utilisation:

\begin{terminal}
>>> dir()
['__annotations__', '__builtins__', '__doc__', '__loader__', '__name__', '__package__', '__spec__']', '__package__', '__spec__']
>>> import tkinter
>>> dir()
['__annotations__', '__builtins__', '__doc__', '__loader__', '__name__', '__package__', '__spec__', 'tkinter']
\end{terminal}


\begin{lstlisting}[language=Python]
dir(name)  # Lister les proprietes/methodes disponibles pour "name"
\end{lstlisting}

Exemple d'utilisation:

\begin{terminal}
>>> dir(tkinter)
['ACTIVE', 'ALL', ...,  'Text', 'Tk', 'TkVersion', ...]
>>> dir(tkinter.Tk)
['_Misc__winfo_getint', '_Misc__winfo_parseitem', '__class__', ..., 'after', 'after_cancel', 'after_idle', 'anchor', 'aspect', 'attributes', 'bbox', ...]
\end{terminal}

% ---------------------------------------------------------------------
\subsubsection{La commande help()}

Elle permet d'accéder à l'aide python. Elle prend un argument, qui correspond à l'objet pour lequel on veut obtenir des informations.
\begin{lstlisting}[language=Python]
help(name)  # Obtenir de l aide sur le module "name"
\end{lstlisting}

Exemple d'utilisation:

\begin{terminal}
>>> help(tkinter.Tk)
>>> help(list)
\end{terminal}

Pour quitter l'aide on peut utiliser la touche 'q' du clavier.

% ---------------------------------------------------------------------
\subsubsection{todo?}

pip list

which python

%%%%%%%%%%%%%%%%%%%%%%%%%%%%%%%%%%%%%%%%%%%%%%%%%%%%%%%%%%%%%%%%%%%%%%%
%                     Environnements virtuels
%%%%%%%%%%%%%%%%%%%%%%%%%%%%%%%%%%%%%%%%%%%%%%%%%%%%%%%%%%%%%%%%%%%%%%%
\subsection{Environnements virtuels}

%%%%%%%%%%%%%%%%%%%%%%%%%%%%%%%%%%%%%%%%%%%%%%%%%%%%%%%%%%%%%%%%%%%%%%%
%                     Listes
%%%%%%%%%%%%%%%%%%%%%%%%%%%%%%%%%%%%%%%%%%%%%%%%%%%%%%%%%%%%%%%%%%%%%%%
\subsection{Listes}

\subsubsection{Accéder à des éléments}

Pour accéder aux derniers éléments d'une liste:

\begin{lstlisting}[language=Python]
ma_liste = [1,2,3,4]
print(ma_liste[-1])  # affiche 4
print(ma_liste[-2])  # affiche 3
\end{lstlisting}

Attention, cette notation aussi peut générer une exception "IndexError".

\subsubsection{Parcourir des listes}

Pour parcourir une liste sans utiliser d'indice
\begin{lstlisting}[language=Python]
ma_liste = [1, 2,"test"]
for val in ma_liste:
    print(val)
\end{lstlisting}
Affiche:
\begin{terminal}
1
2
test
\end{terminal}

Pour parcourir une liste tout en récupérant l'indice
\begin{lstlisting}[language=Python]
ma_liste = [1, 2, 3, 4, "toto", pi]
for idx, val in enumerate(ma_liste):
    print(f"ma_liste[{idx}] = {val}")
\end{lstlisting}
Affiche:
\begin{terminal}
ma_liste[0] = 1
ma_liste[1] = 2
ma_liste[2] = test
\end{terminal}




%%%%%%%%%%%%%%%%%%%%%%%%%%%%%%%%%%%%%%%%%%%%%%%%%%%%%%%%%%%%%%%%%%%%%%%
%                     Lambda fonction
%%%%%%%%%%%%%%%%%%%%%%%%%%%%%%%%%%%%%%%%%%%%%%%%%%%%%%%%%%%%%%%%%%%%%%%
\subsection{Fonctions Lambda / Fonctions sans noms}




